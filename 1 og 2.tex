\documentclass{article}
\usepackage[utf8]{inputenc}
\title{Obligatorisk innlevering 3 - Objektorientert Programmering HiØ}
\author{Ådne Tobiesen}
\date{Februar 2018}
\begin{document}
   \maketitle
   
   \section*{Oppgave 1}
   \begin{itemize}  
    \item Refactor    - Dette vil si at du bytter ut intern informasjon om noe i programmet ditt, denne endringen vil da skje automatisk over hele prosjektet. Et eksempel: refactoring-funksjonen i IDE-en som gjør at du kan bytte både navn og datatype knirkefritt.
    \item Static - Static vil si at metoden eller klassen kan eksistere uavhengig av instansvariablene som blir lagd av klassen. For eksempel: Dersom vi har en klasse som heter deltakere, så kan vi ha en statisk metode som teller antall deltakere.
    \item Abstract - En tom klasse som vi ikke kan bruke for å lage objekter ifra, denne ligger gjerne på toppen av hierarkiet. Denne er nyttig når vi snakker om polymorfisme (Antonym konkret) En abstract-metode er en metode som må bli overridden, og må være inneholdt i en abstract-klasse.
    
    \item Interface - Likner på en klasse, men er ikke dette. Ift arv så går denne på siden av denne i strukturen, og en klasse kan per definisjon ikke arve, men heller implementere denne.
     Innholder en mal for hva en klasse skal inneholde, nærmere bestemt: et grensesnitt. Påtvinger en mal for flere utviklere. Opprinnelig så måtte alt i dette være abstrakt. Implementerer Er greit hvis man skal ha en felles standard på flere ting som i utgpunktet ikke er relaterte. Et eksempel: så kan vi ha en kjøretøy-klasse med en Kjøre-metode, denne implementeres ulikt enten det er en motorsykkel, en båt, et fly eller en bil.Ikke en klasse. 
    \end{itemize}
   \newpage
   \section*{Oppgave 2}
   Samarbeidspartner: Kristine Hemstad
   
   Forskjeller:
   \begin{itemize}
   \item Min er på Engelsk, grunn: jeg er vant med engelsk faglitteratur.
   \item  Min hadde eget episode-objekt på addEpisode, ikke hardkodet. Grunn: Sikkert lite resurseffektivt, men er ryddigere (For min del).
   \item Hadde ikke if-test for å skrive sesong 4, men egen metode. Riktignok så har jeg fjernet mye kode ifra min innlevering så man kan si at entropien har økt.
   
   \item Getterne og setterne er separate på min, mens på Kristine sin så er de annenhver. 
   
   \item Konklusjon: Essensen er, sånn totalt sett, den samme.
   \end{itemize}
\end{document}