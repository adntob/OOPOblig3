\documentclass{article}
\usepackage[utf8]{inputenc}
\title{Obligatorisk innlevering 4 - Objektorientert Programmering HiØ}
\author{Ådne Tobiesen}
\date{Februar 2018}
\begin{document}
   \maketitle
   
   \section*{Oppgave 1}
   
   
   
   \newpage
   \section*{Oppgave 2}
   Samarbeidspartner: Jørn Henning Larsen
   
   Forskjeller:
    Oppgavene innbydde ikke til mye variasjon, så essensen er så og si lik
    Oppgaven hans har flere parametere for person (Kjønn, nasjonalitet...), mens min kun har navn, etternavn og alder.
    map.entry brukes på karakterer i tv-serien, min brukte en poor-mans metode ved å ha dette innebygd i selve serien.
    For-løkkene er ulike på oppgaven om tv-serie, mens jeg har en for-løkke med en variabel i den innerste, så har Jørn en utvidet for-løkke. (Dermed brukes put istedet for add)
  
\end{document}